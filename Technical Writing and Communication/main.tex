% ----------------------------------

% Η εργασία υλοποιήθηκε σε overleaf

% ----------------------------------

\documentclass[12pt]{article}

\usepackage{cite}
\usepackage{graphicx}

\usepackage[cm-default]{fontspec}
\setromanfont{FreeSerif}
\setsansfont{KerkisSans}

\usepackage{xltxtra}
\usepackage{xgreek}
\usepackage{etex}
\setmainfont[Ligatures=Tex]{Kerkis}



\begin{document}
\begin{center}
\includegraphics[width=.7\textwidth]{logo.jpg}
\end{center}

\begin{center}\Large{\texttt{ΤΜΗΜΑ ΜΗΧΑΝΙΚΩΝ ΗΛΕΚΤΡΟΝΙΚΩΝ ΥΠΟΛΟΓΙΣΤΩΝ ΚΑΙ ΠΛΗΡΟΦΟΡΙΚΗΣ}}\end{center}

\begin{center}
    \LARGE{\textbf{Συγγραφή και Παρουσίαση Τεχνικών Κειμένων}} 
\end{center}

\begin{center}
\begin{tabular}{rl}
\textbf{Οναματεπώνυμο(1)}:& \textbf{Μπουρλάκης Γεώργιος} \\
\textbf{ΑΜ(1)}:& \textbf{1054321} \\
\end{tabular}{}
\end{center}
\newpage

\begin{center}
    \Huge{\textbf{ΠΕΡΙΕΧΟΜΕΝΑ}} 
\end{center}
\textbf{1. ΕΠΙΛΟΓΗ ΑΡΘΡΩΝ..........................................  3}\\
\textbf{2. ΠΕΡΙΛΗΨΗ.....................................................  4} \\
\textbf{3. ΑΝΑΣΚΟΠΗΣΗ ΑΡΘΡΩΝ...................................  5} \\
\textbf{4. ΒΙΒΛΙΟΓΡΑΦΙΑ................................................  8} \\
\textbf{5. ΒΙΟΓΡΑΦΙΚΑ ΣΗΜΕΙΩΜΑΤΑ............................  10} 
\newpage

\title{\textbf{<<Επιλογή άρθρων>>}}
\date{April 2020}
\maketitle

\begin{tabular}{ll}

\textbf{ΟΝΟΜΑΤΕΠΩΝΥΜΟ:}& ΜΠΟΥΡΛΑΚΗΣ \\ 
\textbf{SHA-1:}& d00d5fed6aeca0b1b4a6506cad4ccb3a8a9fb488 \\

\textbf{Διακριτά ψηφία:}& d, 0, 5, f, e, 6 \\

\end{tabular}

\newpage
\begin{center}\LARGE{\title{\textbf{<<Περίληψη>>}}}\end{center}
\begin{abstract*}
    Τα κείμενα που μελετήσαμε ασχολούνται κατά κύριο λόγο με έρευνες που σχετίζονται με τον άνθρωπο και τον τρόπο με τον οποίο η τεχνολογία έχει εισβάλλει και επηρεάσει την ζωή του. Πιο αναλυτικά, γίνεται λόγος για έξυπνα συστήματα βασιζόμενα στον τομέα της τεχνητής νοημοσύνης τα οποία χρησιμοποιεί ο άνθρωπος για να αναπτύξει νέες επιστήμες καθώς και την οικονομία. Τέτοια παραδείγματα είναι οι διάφορες εφαρμογές που χρησιμοποιούμε στις μέρες μας, όπως το σύστημα ηλεκτρονικής ομιλίας (chatbot). Σε άλλα άρθρα πληροφορούμαστε για τον τομέα της συνεργατικής μάθησης και τις επικοινωνιακές δεξιότητες των ατόμων, οι οποίες μπορούν να αναπτυχθούν ορθά και αποτελεσματικά με τον κατάλληλο σχεδιασμό μάθησης δηλαδή το σύστημα αλληλεπίδρασης των ατόμων μεταξύ τους και του εκάστοτε εκπαιδευτικού συστήματος. Επιπλέον, γίνεται μία προσπάθεια εφαρμογής του τρόπου σκέψης του ανθρώπου πάνω στα έξυπνα συστήματα, όπου αυτός ο τρόπος σκέψης βασίζεται στην διαίρεση των μεγάλων προβλημάτων σε υποπροβλήματα για ευκολότερη διαχείριση των πρώτων. Όλα αυτά εξελίσσονται μέσα από έρευνες ατόμων σε διάφορα πανεπιστημιακά ιδρύματα ανά τον κόσμο.
\end{abstract*}


\newpage
\begin{center}\LARGE{\title{\textbf{<<Ανασκόπηση άρθρων>>}}}\end{center}
\maketitle
\begin{center}\section*{Η ανάπτυξη  εκπαιδευτικών και πνευματικών δεξιοτήτων του ανθρώπου μέσα από την τεχνολογία και την τεχνογνωσία}\end{center}

\begin{abstract*}
    Τα άρθρα που μας ανατέθηκαν αναφέρονται στην προσπάθεια του ανθρώπου να ενσωματώσει τη σκέψη του σε διάφορα συστήματα με το πέρασμα του χρόνου, καθώς και σε τεχνικές επικοινωνίας μεταξύ ατόμων για τη βέλτιστη συνεργασία.  
    
    Ο αρθρογράφος του [d], φοιτητής σε πανεπιστήμιο που ήταν καθηγητής ο Simon, εμπνέεται από μία φράση του που λέει να μην καταναλώνεσαι με την κληρονομιά των υπόλοιπων, αλλά να περιμένεις μέχρι να ολοκληρωθεί η δική σου. Στη συνέχεια, αναφέρεται στην πρώτη του γνωριμία με την τεχνητή νοημοσύνη και στο ξεκίνημα έρευνας σχετικής με την επίλυση προβλημάτων. Έπειτα, καταλήγει στο συμπέρασμα ότι ένα έξυπνο σύστημα πρέπει συνεχώς να μπορεί να μαθαίνει, να είναι επαρκές και να αντιλαμβάνεται την κοινή αίσθηση των ανθρώπων. Αναφέρε-ται, επίσης, στην εξέλιξη των έξυπνων συστημάτων με σημαντικά ορόσημα στην πορεία του χρόνου. Τονίζει, ακόμα, τη σημασία της σωστής έρευνας και ακριβούς κατανό-ησης προβλημάτων. Υποστηρίζεται, ότι με τη δουλειά και τα πειράματα του Simon εξελίχτηκαν πολλές επιστήμες, ενώ αναπτύχθηκε και η οικονομία. Τέλος, ο αρθρογράφος αναφέρει μια συζήτηση τους, καταλήγοντας ότι δεν ασχολή-θηκε με ποικίλους τομείς, απλά προσπάθησε να αναλύσει το πως παίρνουν αποφάσεις οι άνθρωποι και τον τρόπο σκέψης τους.

   Ο αρθρογράφος στο [0] παρομοιάζει ένα «μαγικό κόλπο» με την επίλυση δύσκολων, για τον άνθρωπο, προβλη-μάτων, καθώς και στι2 δύο περιπτώσεις το κοινό έμεινε άναυδο, εξαιτίας της μη κατανόησης της διαδικασίας. Έπειτα, αναλύεται αυτή η διαδικασία σε κάποια βασικά βήματα, ώστε να γίνει σαφής η εξήγηση. Προγράμματα, όπως του Newell και του Simon, βρίσκουν πληροφορίες, τις συγκρίνουν και καταλήγουν σε λύση προβλημάτων, διαδικασία που διαφέρει από αυτήν της ανθρώπινης επίλυσης. Στη συνέχεια, εξηγείται ένα παράδειγμα άθροισης δύο λέξεων με αποτέλεσμα μία άλλη και με πόσους διαφορετικούς συνδυασμούς μπορεί να υλοποιηθεί. Ακόμα, η διαδικασία της αναζήτησης προχωράει είτε με την περισσότερο υποσχόμενη πληροφορία είτε με την πιο εύκολη και γρήγορη. Ύστερα, αναφέρονται διάφορες πηγές από τις οποίες δημιουργούνται δειγματοχώροι για ποικίλα προβλήματα. Τέλος, αναδεικνύεται η ομορφιά στο διαχωρισμό μεγάλων και πολύπλοκων προβλημάτων σε μικρά και απλά, που με το συνδυασμό τους το αποτέλεσμα φαίνεται μεγαλειώδες στους υπόλοιπους, όπως με τα «μαγικά» κόλπα. 

  Στο [5] ο αρθρογράφος αναφέρεται εφαρμογών και διεπαφών στα κινητά τηλέφωνα, καθώς αυξάνεται η καθημε-ρινή χρήση τους όλο και περισσότερο. Στη συνέχεια, παρουσιάζονται διάφορες έρευνες για τον τρόπο με τον οποίο χειρονομίες κρατώντας το κινητό είναι αποδεκτές και χρήσιμες στους ανθρώπους, προκειμένου να γίνονται επιθυμητές ενέργειες. Έτσι, προέκυψαν στατιστικά τα οποία διέφεραν ανάλογα την τοποθεσία του χρήστη, αν βρίσκεται στη δουλειά, στο σπίτι κλπ, την περιοχή που ζει, Αγγλία ή Αμερική, και το αν ο χρήστης είναι μόνος ή με παρέα ή με κάποιον άγνωστο. Έπειτα, ερευνήθηκε οι λόγοι που κάποιες χειρονομίες άρεσαν και άλλες όχι. Παρατηρήθηκε ότι τους άρεσαν χειρονομίες με μικρές, γνωστές, καθημερινές κινήσεις, ενώ δεν τους άρεσαν άλλες με περίεργες κινήσεις που δεν τις έκαναν καθημερινά.  Τέλος, οι έρευνες θα συνεχιστούν αναζητώντας νέα μοτίβα που θα κάνουν αποδεκτές τις χειρονομίες, αφού είναι ένα θέμα που εξαρτάται από πολλούς παράγοντες.

  Ο κειμενογράφος του [f], επιλέγει να ασχοληθεί με το κομμάτι της συνεργατικής μάθησης, το οποίο ορίζει ως την προσπάθεια μεταξύ δύο ή περισσότερων ατόμων με σκοπό την μάθηση. Στην συνέχεια, επεκτείνεται στον τρόπο με τον οποίο η ποικιλία των κλιμάκων επηρεάζει το παραπάνω ζήτημα και εντοπίζει διάφορα χαρακτηριστικά, όπως ότι η έννοια της ατομικής μνήμης είναι πιο χρήσιμη από αυτή της ομαδικής. Όλα αυτά εφαρμόζονται σε ένα πλαίσιο όπου οι συνεργάτες πρέπει να διακατέχονται από υπομονή και κατανόηση, κάνοντας χρήση κυρίως της σκέψης, δηλαδή του διαλόγου με τον εαυτό μας ώστε να μπορέσουμε να κάνουμε διάλογο με τον συνεργάτη μας. Εν συνεχεία, αναφέρει τη σημασία της μάθησης και τη συνδέει με τη σημασία της συνεργασίας μιλώντας για τους τρόπους αλληλεπίδρασης των δύο. Κλείνοντας, κάνει λόγο για θεωρίες της συνεργατικής μάθησης και το πως αυτή εφαρμόζεται σε νέες μελέτες που μπορεί να προκύψουν.

  Στο κείμενο [e], γίνεται αντιληπτό το ενδιαφέρον του συγγραφέα σχετικά με τον σχεδιασμό της μάθησης. Πιο συγκεκριμένα μέσα από διάφορες μελέτες κυρίως από το Open University στο Ηνωμένο Βασίλειο βλέπουμε τον τρόπο με τον οποίο μελετώντας την ανάλυση της μάθησης οδηγούμαστε στο σχεδιασμό της. Παρατηρώντας τα αποτελέσματα αυτών των ερευνών, μπορούμε να κατανοήσουμε σαφέστερα την αλληλεπίδραση των μαθητευόμενων με την σχεδιασμένη μάθηση που τους επιβάλλεται αλλά και το πως η επικοινωνία και η συνεργασία μεταξύ τους επηρεάζει την μάθηση. Εκτενέστερα, γίνεται λόγος για τις δραστηριότητες της μάθησης οι οποίες συμβάλλουν στην ανάπτυξη δεξιοτήτων των μαθητών μέσω της εμπειρίας και του πειραματισμού των τελευταίων. Ανακεφαλαιώνοντας, με το πέρα-σμα του χρόνου και των ερευνών βλέπουμε τα αποτελέσματα κάθε συστήματος και κάθε λειτουργίας της μάθησης.

  Το άρθρο [6] μας δίνει πληροφορίες σχετικά με μία έρευνα που έγινε από τον Sameera A. Abdul από το Essex University του Ηνωμένου Βασιλείου. Η έρευνα αυτή αφορά τον τεχνικό σχεδιασμό συστημάτων ομιλίας στους ηλεκτρονικούς υπολογιστές και την αλληλεπίδραση τους με τον άνθρωπο. Πιο συγκεκριμένα, μας ορίζει τον όρο “Chatbot” 	και μας εξηγεί διάφορες τεχνικές υλοποίησης του την τελευταία δεκαετία. Επιπλέον, γίνεται λόγος για το πόσο σημαντικό μέσο επικοινωνίας είναι ο διάλογος και τις προσπάθειες που πραγματοποιούνται για την καλύτερη προσαρμογή του στα ηλεκτρονικά μέσα. Προκειμένου, ο υπολογιστής να μπορέσει να καταλάβει τον άνθρωπο σε μεγαλύτερο βαθμό χρησιμοποιούνται διάφορα εργαλεία όπως είναι οι γλώσσες προγραμματισμού: JavaScript, Python, SQL κλπ.
  
  Τέλος, μπορούν να μελετηθούν και άλλες ενδιαφέρουσες πηγές [1], [2], [3], [4], [7], [8], [9], [a], [b], [c] για το θέμα.
\end{abstract*}
\newpage

\begin{center}\LARGE{\title{\textbf{<<Βιβλιογραφία>>}}}\end{center}


\bibliographystyle{IEEEtran}
\bibliography{bibliography.bib}
\cite{simon1971human}
\cite{clark2019state}
\cite{simon1995artificial}
\cite{arefin2016exploring}
\cite{koelle2018acceptable}
\cite{rico2010usable}
\cite{abdul2015survey}
\cite{ferguson2012learning}
\cite{siemens2013learning}
\cite{koelle2019acceptable}
\cite{gavsevic2015let}
\cite{dillenbourg2002over}
\cite{dillenbourg2010technology}
\cite{feigenbaum2013hath}
\cite{rienties2016impact}
\cite{dillenbourg1999you}
\newpage

\begin{center}\LARGE{\title{\textbf{<<Βιογραφικά Σημειώματα>>}}}\end{center}

\end{document}
